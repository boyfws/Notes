\documentclass{article}
\usepackage{amsmath} % Для математических формул
\usepackage{enumitem} % Для настройки списков
\usepackage{tikz} % Для диаграмм Эйлера
\usepackage[utf8]{inputenc}
\usepackage[russian]{babel} % Для поддержки русского языка

\title{Макроэкономика 1}
\author{Минкин Даниэль}

\begin{document}

\maketitle

\tableofcontents % Оглавление


\section{Введение}

\subsection{Формула оценки}

\begin{equation}
    0.2 \cdot \text{Ср1т} + 0.2 \cdot \text{Ср2т} + 0.25 \cdot \text{Кр1} + 0.35 \cdot \text{Кр2}
\end{equation}
где:
\begin{itemize}
    \item $\text{Ср1т}$ --- Ср 1 типа (верно-неверно) средняя за 5 работ
    \item $\text{Ср2т}$ --- Ср 2 типа (ответ на вопрос с пояснением) средняя за 5 работ
    \item $\text{Кр1}$ --- Контольная работа 1
    \item $\text{Кр2}$ --- Контольная работа 2
\end{itemize}


\section{Лекция 1}

\subsection{Макроэкономические агенты}

\begin{itemize}
    \item Домашнее хозяйство
    \item Фирмы (бизнес)
    \item Государство
    \item Иностранный сектор 
\end{itemize}

\subsection{Формулы}

Домашнее хозяйство тратит располагаемый доход на потребление и сбережение 

\begin{equation}
    Y_{d} = Y - T
\end{equation}
где $Y_{d}$ --- располагаемый доход, $Y$ --- общий доход, а $T$ --- чистые налоги

\quad

\begin{equation}
    Y_{d} = C + S
\end{equation}
где $C$ --- потребление, $S$ --- сбережение

\quad

\begin{equation}
    T = T_{X} - T_{R}
\end{equation}
где $T_{X}$ --- налоги, $T_{R}$ --- трансферты

\quad

\begin{equation}
    T_{X} = T_{X_{0}} + t * Y
\end{equation}
где $t$ --- налоговая ставка 

\quad

Из этого легко вывести следующее:

\quad

\begin{gather*}
    Y - T = C + S \\
    \Rightarrow Y - T_{X} + T_{R} = C + S \\
    \Rightarrow Y - T_{X_{0}} - t * Y + T_{R} = C + S
\end{gather*}

\subsection{Открытые и закрытые экономики}

Экономики разделяют на открытые и закрытые по принципу наличия иностранного сектора как экономического агента. В случае если он присутствует, выполняется уравнение:

\begin{equation}
    N_{X} = Ex - Im
\end{equation}
где $N_{X}$ --- чистый экспорт, $Ex$ - экспорт, $Im$ - импорт 

\subsection{ВВП и ВНП}

\begin{center}
    \begin{tikzpicture}
        \draw[black, thick] (0,0) circle (2cm);
        \draw[black, thick] (2,0) circle (2cm);
    
        % Добавляем метки в непересекающиеся и пересекающиеся области
        \node at (-1, 0) {1}; % Левая часть первого круга
        \node at (3, 0) {3};  % Правая часть второго круга
        \node at (1, 0) {2};  % Пересечение двух кругов
    
        % Добавляем подписи к кругам
        \node[black] at (-1.5, 2.2) {ВВП};
        \node[black] at (3.5, 2.2) {ВНП};  
    \end{tikzpicture}
\end{center}

\begin{itemize}
    \item 1 --- Производим на нашей территории, не наши факторы.
    \item 2 --- Наша страна, свои факторы.
    \item 3 --- Другая страна, наши факторы.
\end{itemize}


\begin{equation}
    \text{ВВП}_{\text{по расходам}} = C + I + G + N_{X}
\end{equation}
где $C$ --- доходы, $I$ --- инвестиции фирм, $G$ --- гос. закупки, $N_{X}$ --- иностранный сектор (чистый экспорт)





















\end{document}