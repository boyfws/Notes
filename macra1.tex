\documentclass{article}
\usepackage{amsmath} % Для математических формул
\usepackage{enumitem} % Для настройки списков
\usepackage{tikz} % Для диаграмм Эйлера
\usepackage[utf8]{inputenc}
\usepackage[russian]{babel}
\usepackage{tcolorbox}
\usepackage{wasysym} % Для поддержки русского языка

\title{Макроэкономика 1}
\author{Минкин Даниэль}

\begin{document}

    \maketitle

    \tableofcontents % Оглавление


    \section{Введение}

    \subsection{Формула оценки}

    \begin{equation}
        0.2 \cdot \text{Ср1т} + 0.2 \cdot \text{Ср2т} + 0.25 \cdot \text{Кр1} + 0.35 \cdot \text{Кр2}
    \end{equation}
    где:
    \begin{itemize}
        \item $\text{Ср1т}$ --- Ср 1 типа (верно-неверно) средняя за 5 работ
        \item $\text{Ср2т}$ --- Ср 2 типа (ответ на вопрос с пояснением) средняя за 5 работ
        \item $\text{Кр1}$ --- Контольная работа 1
        \item $\text{Кр2}$ --- Контольная работа 2
    \end{itemize}


    \section{Лекции 1-2}

    \subsection{Макроэкономические агенты}

    \begin{itemize}
        \item Домашнее хозяйство
        \item Фирмы (бизнес)
        \item Государство
        \item Иностранный сектор
    \end{itemize}

    \subsection{Формулы}

    Домашнее хозяйство тратит располагаемый доход на потребление и сбережение

    \begin{equation}
        Y_{d} = Y - T
    \end{equation}
    где $Y_{d}$ --- располагаемый доход, $Y$ --- общий доход, а $T$ --- чистые налоги

    \quad

    \begin{equation}
        Y_{d} = C + S
    \end{equation}
    где $C$ --- потребление, $S$ --- сбережение

    \quad

    \begin{equation}
        T = T_{X} - T_{R}
    \end{equation}
    где $T_{X}$ --- налоги, $T_{R}$ --- трансферты

    \quad

    \begin{equation}
        T_{X} = T_{X_{0}} + t * Y
    \end{equation}
    где $t$ --- налоговая ставка

    \quad

    Из этого легко вывести следующее:

    \quad

    \begin{gather*}
        Y - T = C + S \\
        \Rightarrow Y - T_{X} + T_{R} = C + S \\
        \Rightarrow Y - T_{X_{0}} - t * Y + T_{R} = C + S
    \end{gather*}

    \subsection{Открытые и закрытые экономики}

    Экономики разделяют на открытые и закрытые по принципу наличия иностранного сектора как экономического агента. В случае если он присутствует, выполняется уравнение:

    \begin{equation}
        N_{X} = Ex - Im
    \end{equation}
    где $N_{X}$ --- чистый экспорт, $Ex$ - экспорт, $Im$ - импорт

    \subsection{ВВП и ВНП}

    \begin{quote}
        \textbf{ВВП} --- суммарная рыночная стоимость всех конечных товаров
        и услуг произведенных на территории страны в течении года. Т.е
        мы смотрим на сумму стоимостей всех товаров, которые произвели у нас в стране, как с помощью
        наших факторов производства, так и на наших, важно, что это произведено именно у нас в стране
    \end{quote}

    \begin{quote}
        \textbf{ВНП} --- суммарная стоимость всех конечных товаров и услуг
        произведенных с помощью наших факторов производства за год. Т.е нам не важно
        где мы, что-то произвели, нам важно, что мы произвели это с помощью наших факторов
        производства
    \end{quote}

    Если $\text{ВВП} > \text{ВНП}$ то в нашей стране 100\% есть иностранные производители, так как даже если
    мы представим, что все наши факторы производства используются в нашей стране (что не так в общем случае),
    то все равно останется излишек, который объясняется иностранными факторами производства. Однако, если
    $\text{ВВП} < \text{ВНП}$ иностранные факторы производства могут как быть, так и не быть в нашей стране. Это
    связано с тем, что ВНП, говорит нам только о наших факторах производства, без привязки к локации. Так --- весь
    наш ВНП может быть получен заграницей, а весь ВВП получен от иностранных факторов производства

    \begin{center}
        \begin{tikzpicture}
            \draw[black, thick] (0,0) circle (2cm);
            \draw[black, thick] (2,0) circle (2cm);

            % Добавляем метки в непересекающиеся и пересекающиеся области
            \node at (-1, 0) {1}; % Левая часть первого круга
            \node at (3, 0) {3};  % Правая часть второго круга
            \node at (1, 0) {2};  % Пересечение двух кругов

            % Добавляем подписи к кругам
            \node[black] at (-1.5, 2.2) {ВВП};
            \node[black] at (3.5, 2.2) {ВНП};
        \end{tikzpicture}
    \end{center}

    \begin{itemize}
        \item 1 --- Производим на нашей территории, не наши факторы.
        \item 2 --- Наша страна, свои факторы.
        \item 3 --- Другая страна, наши факторы.
    \end{itemize}

    \textbf{Что мешает оценке совокупной рыночной стоимости?}
    \begin{itemize}
        \item Не все товары продаются на рынке (например, я произвел фигурку из дерева и подарил соседу --- я внес вклад в экономику, однако его сложно оценить денежно)
        \item Не все товары продаются по рыночным ценам (например, мои дочерние фирмы продают друг другу товары по мизерным ценам --- оценка товаров по такой цене не справедлива)
        \item Есть теневая экономика (например, я гоняю порши через границу и продаю их в темную)
    \end{itemize}

    \subsection{Виды товаров}

    \begin{quote}
        \textbf{Товары разделяются на два вида --- конечные и промежуточные}
    \end{quote}

    \begin{quote}
        \textbf{Промежуточные товары} --- товары использованные в производстве конечных товаров
        в исследуемый промежуток времени (Последняя приписка очень важна --- если мы закупили материалы
        под конец года, но не использовали их в производстве, считается, что мы являемся конечными потребителями
        материалов)
    \end{quote}

    \begin{quote}
        \textbf{Конечные товары} --- товары, потребленные в исследуемый промежуток времен
    \end{quote}

    Можно заметить, что стоимость конечного товара учитывает в себе стоимость промежуточных товаров, так как
    на каждом этапе, поставщик промежуточных товаров пытается продать свой товар, выше цены использованных им
    промежуточных товаров, исходя из своей экономической выгоды

    \subsection{Стоимость конечной продукции}

    \begin{quote}
        \textbf{Добавленная стоимость фирмы} --- это выручка за вычетом стоимости использованных ей промежуточных товаров. I.e
        это понятие очень похожее на прибыль фирмы. Мы взяли выручку и вычли затраты на то, что мы использовали в производстве (т.е издержки на
        производство)
    \end{quote}

    Заметим, что цена конечного товара может быть выражена как сумма добавленных стоимостей i.e прибылей фирм


    \subsection{Чистый национальный продукт}

    \begin{equation}
        \text{Чистый внутренний продукт} = \text{ВВП} - \text{амортизация}
    \end{equation}

     \begin{equation}
        \text{Чистый национальный продукт} = \text{ВНП} - \text{амортизация}
    \end{equation}

    \textbf{Почему мы вычитаем амортизацию?}

    \quad

    Амортизация условно отражает, насколько износилось оборудование в ходе производства за год. В экономической теории
    считается, что оборудование ``переносит`` свою стоимость на товар, т.е ``новой`` ценностью произведенной в экономике является
    стоимость товара за вычетом той стоимости, которая была перенесена в него из оборудования

    \quad

    \textbf{Почему стоимость оборудования, которая была перенесена нельзя назвать новой?}

    \quad

    Дело, в том, что стоимость оборудования уже была учтена до этого в ВВП/ВНП в тот момент, когда оно
    было продано/произведено. Т.е стоимость обрудования перенесенная на продукт, будет по факту учтена дважды --- тогда,
    при его производстве/продаже и сейчас внутри товара. Именно поэтому для наблюдаения полной картины, нужно вычесть амортизацию из ВВП/ВНП

    \subsection{Факторный доход}

    \begin{quote}
        \textbf{Факторный доход} --- сумма доходов, заработанных факторами производства
        в собственности граждан страны
    \end{quote}

    Компоненты факторного дохода
    \begin{itemize}
        \item Заработная плата или жалованье
        \item Рентные платежи
        \item Процентные платежи
        \item Прибыль
    \end{itemize}


    \subsection{Рассчет ВВП}
    Выделяют три метода:

    \begin{itemize}
        \item Производственный метод (по добавленной стоимости)
        \item Метод конечного использования (по расходам)
        \item Распределительный метод (по доходам)
    \end{itemize}

    Метод расчета по добавленной стоимости уже показывался выше. Рассмотрим и другие

    \quad

    \textbf{В теории все подходы к подсчету ВВП должны давать одинаковый результат (На практике не так)}


    \subsubsection{Расчет ВВП по расходам}

    \begin{equation}
        \text{ВВП}_{\text{по расходам}} = C + I + G + N_{X}
    \end{equation}
    где $C$ --- потребительские расходы, $I$ --- инвестиции фирм, $G$ --- гос. закупки, $N_{X}$ --- иностранный сектор (чистый экспорт)

    На практике, потреб. расходы вносят 2/3 вклада в ВВП

    \quad

    Важно заметить, что покупка ценных бумаг не включается в инвестиционные расходы, так как это перераспределение благ, а не их создание

    \quad

    \paragraph{Что считается инвестиционными расходами?}

    \begin{itemize}
        \item Покупка нового оборудования
        \item Новое промышленное строительство
        \item Новое жилищное строительство
        \item Инвестиции в запасы
    \end{itemize}

    \quad

    \textbf{Инвестиции в запасы} --- чистые изменения запасов фирм

    \quad

    \textbf{Почему мы берем именно дельту запасов?}

    \quad

    Во-первых, для наглядности можно представить, что компания не делала
    ничего в течение года. Если мы будем считать запасы по их обычной стоимости мы дважды учтем одни и те же запасы в первый и второй год.

    Во-вторых, использование дельты помогает предотвращать дублирование учета запасов, так как запасы прошлого года могут быть проданы и учтены в потреблении
    в ВВП текущего года.

    В-третьих, интуитивно это можно объяснить так: Представим, что все запасы прошлого года полностью продаются, при этом все запасы, которые
    есть на конец текущего года - это новая произведенная продукция. Тогда следуя пункту два, мы вычитаем из полученного ВВП запасы прошлого года, и добавляем запасы текущего
    так как они были произведены в этом году --- в итоге получаем дельту
    
    \paragraph{Виды инвестиций}

    Инвестиции делятся на два вида:
    \begin{itemize}
        \item \textbf{Восстановительные инвестиции} --- инвестиции для замены и восстановления капитала
        \item \textbf{чистые инвестиции} --- инвестиции, которые делаются для создания новых факторов производства (в отличие от восстановительных, которые \textbf{поддерживают} старые факторы производства)
        \item \textbf{Валовые инвестиции} --- сумма первых двух видов
    \end{itemize}

    Например, купить новый станок на завод --- восстановительная инвестиция, а построить новый завод --- чистая

    \begin{equation}
        I_{\text{вал.}} = I_{\text{восст.}} + I_{\text{чист.}}
    \end{equation}
    
    \subsubsection{Рассчет ВВП по доходам}
    

    \begin{multline}
        \text{ВВП} = \text{Факторный доход} + \text{Амортизация} + \text{Косвенные налоги} \\
        - \text{Субсидии} - \text{Чистый факторный доход из-за границы}
    \end{multline}


    \textbf{Почему формула выглядит именно так?}

    \quad

    \begin{itemize}
        \item \textbf{Амортизация} --- она вычитается из прибыли фирм в ФД, поэтому нужно ее добавить
        \item \textbf{Косвенные налоги} --- они включаются в цены, при покупке но не участвуют в факторном доходе
        \item \textbf{Субсидии} --- являются частью доходов фирм, но не включаются в расходы
    \end{itemize}
    
    \subparagraph{По поводу чистого факторного дохода из-за границы}
    
    \begin{multline}
        \text{ВНД} = \text{ВВП} \\
        + \text{Доходы резидентов за границей}
        - \text{Доходы нерезидентов в стране}
    \end{multline}

    \quad

    При этом:

    \quad

    \begin{multline}
       \text{Чистый факторный доход из-за границы} = \\
       \text{Доходы резидентов за границей} - \text{Доходы нерезидентов в стране}
    \end{multline}

    \quad

    Формула без чистого факторного дохода из-за границы представляет собой расчет ВНП,
    так как факторный доход учитывает, факторы производства \textbf{в собственности граждан страны}, следовательно
    чтобы получить ВВП нам нужно вычесть данное слагаемое (см. формулу выше)

    \subsection{Реальный и номинальный ВВП}

    \begin{quote}
        \textbf{Номинальный ВВП} --- это ВВП измеренный в ценах \textbf{текущего} года. Т.е мы взяли все товары,
        посмотрели их цены на данный момент и просто посчитали сумму произведенного
    \end{quote}

    \begin{quote}
        \textbf{Реальный ВВП} --- это ВВП измеренный в ценах \textbf{базового} года.
        Т.е мы взяли некий год, который называем базовым записали цены в этот год.
        Теперь когда мы считаем ВВП, мы для каждого товара ищем цену базового года
        и умножаем кол-во произведенных товаров именно на нее при подсчете ВВП
    \end{quote}


    \subsection{Фактический и номинальный выпуск}

    \begin{quote}
        \textbf{Фактический реальный ВВП} ($Y$) --- обычный ВВП, который измеряет ежегодный объем выпуска.
        Характеризует экономический цикл и используется для краткосрочного периода.
    \end{quote}

    \begin{quote}
        \textbf{Потенциальный реальный ВВП} ($Y^{*}$) --- характеризует выпуск в долгосрочном периоде.
        Является характеристикой экономического роста
    \end{quote}

    \subsection{Экономический цикл}


    \begin{quote}
        \textbf{Экономический цикл} --- периодические колебания деловой активности и совокупного реального выпуска
    \end{quote}

    Выделяют следующие последовательные фазы в экономическом цикле

    \begin{itemize}
        \item Подъем
        \item Пик
        \item Спад
        \item Дно
    \end{itemize}

    \subsection{Экономический рост}

    \begin{quote}
        \textbf{Экономический рост} --- долгосрочная тенденция увеличения реального ВВП
    \end{quote}

    \subsection{Темп роста ВВП}

    Темп роста ВВП рассчитывается по следующей формуле:

    \begin{equation}
        g = \dfrac{Y_{t} - Y_{t-1}}{Y_{t-1}} * 100\%
    \end{equation}
    где $Y_{i}$ --- реальный ВВП в год i

    \subsection{Разрыв ВВП}

    Разрыв ВВП рассчитывается по следующей формуле

    \begin{equation}
        gap = \dfrac{Y - Y^{*}}{Y^{*}} * 100\%
    \end{equation}

    Разрыв ВВП позволяет определить в какой стадии экономического цикла находится экономика сейчас ---
    если $gap > 0$ это говорит о периоде подъема/пика в экономике, а если $gap < 0$ то экономика находится в стадии
    спада или дна (a.k.a рецессии)

    \subsection{Индексы цен}

    Индексы цен представляют собой отношение объема выпуска в текущий год и базовый, где под объемом выпуска понимается денежное выражение произведенных товаров.
    В числителе используются цены текущего года, а в знаменателе --- базового

    \begin{equation}
        \text{ИЦ} = \dfrac{\sum P_{i} * q_{i}}{\sum P_{i}^{\text{баз}} * q_{i}}
    \end{equation}

    Подход к выбору $q$ зависит от типа индекса, выделяют два типа:

    \begin{itemize}
        \item Индекс Пааше --- используются количества текущего года
        \item Индекс Ласпераса --- используются кооличества из базового года 

    \end{itemize}
    
    
    У них также есть другие названия: 
    
    \begin{itemize}
        \item Индекс Пааше --- Дефлятор ВНП
        \item Индекс Ласпераса --- Индекс потребительских цен
    \end{itemize}
    
    
    \subsection{Инфляция}
    
    \begin{quote}
        \textbf{Инфляция} --- это устойчивая тенденция повышения
        общего уровня цен, приводящая к потери
        покупательной способности
    \end{quote}
    
    Уровень инфляции оценивается с помощью индекса цен.
    
    Путь $P_{t}$ - некий индекс цен в момент времени t 
    
    Тогда уровень инфляции за период времени 1, определяется как: 
    
    \begin{equation}
        \dfrac{P_{t} - P_{t-1}}{P_{t-1}} * 100\%
    \end{equation}
    
    
    \subsection{Оценка роста номинального ВВП}
    
    Рост номинального ВВП происходит по двум причинам: 
    
    \begin{itemize}
        \item Инфляция 
        \item Рост производства
    \end{itemize}
    
    \begin{equation}
        \text{Темп роста ном. ВВП} = \text{Уровень инфляции} + \text{Темп роста реал. ВВП}
    \end{equation}


    \subsection{Макроэконмические рынки}

    Выделяют следующие макроэкономические рынки:

    \begin{itemize}
        \item Рынок товаров и услуг
        \item Рынок факторов производства
        \item Финансовый рынок
        \item Валютный рынок
    \end{itemize}
    
    \subsubsection{Рынок товаров и услуг}

    Ключевые моменты:

    \begin{itemize}
        \item На рынке функционирует универсальное благо
        \item Цена данного блага - уровень цен
        \item Объем производства данного блага - агрегированный уровень производства (например ВВП)
        \item Совокупный спрос на таком рынке представлен как сумма расходов всех макроагентов
        \item Модель рынка благ различна для разных экономических школ
    \end{itemize}


    \subsubsection{Рынок труда}

    Ключевые моменты:

    \begin{itemize}
        \item На рынке функционирует универсальный труд (человеко-часы) $L$ --- как в микре
        \item Цена --- реальная заработная плата ($w / P$)
        \item Совокупный спрос на труд предъявляют фирмы
        \item Труд предоставляют домохозяйства
    \end{itemize}

    \subsubsection{Финансовый рынок}

    Начнем с разделения финансовых активов, на два класса:

    \begin{itemize}
        \item Денежные активы ($M$)
        \item Неденежные активы: акции ($E$) и облигации ($B$)
    \end{itemize}

    Условие равновесия вытекает из выполнения равновесия для обоих классов активов, следовательно:


    \[
    \begin{cases}
    \text{Спрос на деньги} = \text{Предложение денег} \\
    \text{Спрос на ЦБ} = \text{Предложение ЦЬ}
    \end{cases}
    \]
    где ЦБ --- ценные бумаги (акции, облигации)

    \quad
    
    \textbf{В базовых макроэкономических моделях предполагается,
        что облигации выпускает только гос-во.
        Выпущенные облигации делятся между центральным
        бакном и другими потребителями облигаций}

    \subsubsection{Валютный рынок национальной валюты}

    Важно заметить, что рынок валюты может быть рассмотрен и как рынок нац. валюты, так и как рынок иностранной валюты

    \quad

    Ключевые моменты:

    \begin{itemize}
        \item Рынок совершенно конкурентный
        \item Равновесное состояние достигается, когда спрос на валюту равен ее предложению
        \item Рынок саморегулируется приходя к равновесию
    \end{itemize}
    
    \quad
    
    \textbf{Далее мы будем считать, что спрос на отечественную валюту предъявляют только нерезиденты}

    \subsection{Виды валютных курсов}

    Выделяют два вида валютных курсов:

    \begin{itemize}
        \item Прямой валютный курс ($E^{f / d}$) --- стоимость нашей валюты в иностранной
        \item Обратный валютный курс ($E^{d / f}$) --- стоимость иностранной валюты в отечественной
    \end{itemize}

    \quad

    \textbf{Валютный курс в макроэкономике определяется как отношение индексов цен двух стран}


































\end{document}