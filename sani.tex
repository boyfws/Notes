

% Preamble
\documentclass{article}

% Packages
\usepackage{amsmath}
\usepackage{enumitem} % Для настройки списков
\usepackage{amssymb} % Подключаем пакет для знака следствия
\usepackage[utf8]{inputenc}
\usepackage[russian]{babel}
\usepackage{array}


\allowdisplaybreaks


\title{САНИ}
\author{Минкин Даниэль}

% Document
\begin{document}

    \maketitle

    \tableofcontents % Оглавление

    \section{Введение}

    \subsection{Формула оценки}

    \begin{equation}
        0.1 * \text{Квизы} + 0.4 * \text{Дз} + 0.5 * \text{Экзамен}
    \end{equation}

    Курс включает в себя 4 дз

    Квизы проводятся на лекциях и в конце семинаров

    \section{Лекция 1}

    \subsection{Виды шкал}

    Выделяют следующие виды шкал:

    \begin{itemize}
        \item Номинальная шкала
        \item Порядковая шкала
        \item Интервальная шкала
        \item Шкала разностей
        \item Шкала отношений
        \item Абсолютная шкала
    \end{itemize}

    \begin{quote}
        \textbf{Шкала измерения} --- гомоморфное отображение множества элементов системы с отношениями в множество с заданными логическими отношениями
    \end{quote}

    Вспомним, что такое гомоморфное отношение 
    
    \begin{quote}
        \textbf{Гомоморфное отношение (отображение)} --- отображение сохраняющее свойства заданные на первом множестве, после его отображения в новое.
    \end{quote}

    Заскочим вперед ради примера: допустим у нас есть результаты опросов с двумя вариантами ответов --- ``хорошо`` и ``плохо``. На этом множестве может быть
    задана операция сравнения, а именно $\text{``хорошо``} > \text{``плохо``}$, тогда при переходе к шкале мы должны перейти к порядковой шкале, такой, что
    $f(\text{``хорошо``}) > f(\text{``плохо``})$, например обозначать ``хорошо`` как 1, а ``плохо`` как 0

    Рассмотрим подробнее свойства шкал, которые мы будем рассматривать далее:

    \begin{itemize}
        \item \textbf{Тождество} --- на множестве элементов шкалы задана операция равенства
        \item \textbf{Порядок} --- на множестве элементов шкалы задана операция
        \item \textbf{Нулевая точка} --- точка, с которой начинается отсчет в шкале
        \item \textbf{Единицы измерения} --- no comments
        \item \textbf{Операция сложения, вычитания} --- не нуждается в комментариях))
        \item \textbf{Операция деления, умножения} --- так же не нуждается
        \item \textbf{Допустимое преобразование} --- какое преобразование мы можем выполнить с элементами шкалы, оставаясь в ней же
        \item \textbf{Мода} --- задана ли на множестве элементов шкалы операция поиска моды
        \item \textbf{Медиана} --- задана ли на множестве элементов шкалы операция поиска медианы
        \item \textbf{Ср. арифм и хар-ки рассеяния} --- задана ли на множестве элементов шкалы операции поиска среднего и характеристик разброса
    \end{itemize}

    \subsubsection{Номинальная шкала}

    \begin{itemize}
        \item \textbf{Тождество} --- Да
        \item \textbf{Порядок} --- Нет
        \item \textbf{Нулевая точка} --- Нет
        \item \textbf{Единицы измерения} --- Нет
        \item \textbf{Операция сложения, вычитания} --- Нет
        \item \textbf{Операция деления, умножения} --- Нет
        \item \textbf{Допустимое преобразование} --- Взаимно-однозначное
        \item \textbf{Мода} --- Да
        \item \textbf{Медиана} --- Нет
        \item \textbf{Ср. арифм и хар-ки рассеяния} --- Нет
    \end{itemize}

    \textbf{Примеры:} пол, номера паспортов, ИНН


    \subsubsection{Порядковая шкала}

    \begin{itemize}
        \item \textbf{Тождество} --- Да
        \item \textbf{Порядок} --- Да
        \item \textbf{Нулевая точка} --- Может существовать
        \item \textbf{Единицы измерения} --- Нет
        \item \textbf{Операция сложения, вычитания} --- Нет
        \item \textbf{Операция деления, умножения} --- Нет
        \item \textbf{Допустимое преобразование} --- Строго монотонное
        \item \textbf{Мода} --- Да
        \item \textbf{Медиана} --- Да
        \item \textbf{Ср. арифм и хар-ки рассеяния} --- Нет
    \end{itemize}

    \textbf{Примеры:} оценки успеваемости, рейтинг облигаций



    \subsubsection{Интервальная шкала}

    \begin{itemize}
        \item \textbf{Тождество} --- Да
        \item \textbf{Порядок} --- Да
        \item \textbf{Нулевая точка} --- Опционально, но да
        \item \textbf{Единицы измерения} --- Опционально, но да
        \item \textbf{Операция сложения, вычитания} --- Да
        \item \textbf{Операция деления, умножения} --- Нет
        \item \textbf{Допустимое преобразование} --- Линейное
        \item \textbf{Мода} --- Да
        \item \textbf{Медиана} --- Да
        \item \textbf{Ср. арифм и хар-ки рассеяния} --- Да
    \end{itemize}

    \textbf{Примеры:} Шкала Цельсия



    \subsubsection{Шкала разностей}

    \begin{itemize}
        \item \textbf{Тождество} --- Да
        \item \textbf{Порядок} --- Да
        \item \textbf{Нулевая точка} --- Опционально, да
        \item \textbf{Единицы измерения} --- Однозначно определены
        \item \textbf{Операция сложения, вычитания} --- Да
        \item \textbf{Операция деления, умножения} --- Нет
        \item \textbf{Допустимое преобразование} ---Сдвиг
        \item \textbf{Мода} --- Да
        \item \textbf{Медиана} --- Да
        \item \textbf{Ср. арифм и хар-ки рассеяния} --- Да
    \end{itemize}

    \textbf{Примеры: } Время


    \subsubsection{Шкала отношений}


    \begin{itemize}
        \item \textbf{Тождество} --- Да
        \item \textbf{Порядок} --- Да
        \item \textbf{Нулевая точка} --- Однозначно определена
        \item \textbf{Единицы измерения} --- Опционально, но да
        \item \textbf{Операция сложения, вычитания} --- Да
        \item \textbf{Операция деления, умножения} --- Да
        \item \textbf{Допустимое преобразование} --- Подобие $f(x) = a * x$
        \item \textbf{Мода} --- Да
        \item \textbf{Медиана} --- Да
        \item \textbf{Ср. арифм и хар-ки рассеяния} --- Да
    \end{itemize}

    \textbf{Примеры: } шкала Кельвина, масса тела и длина

    \subsubsection{Абсолютная шкала}


    \begin{itemize}
        \item \textbf{Тождество} --- Да
        \item \textbf{Порядок} --- Да
        \item \textbf{Нулевая точка} --- Однозначно определена
        \item \textbf{Единицы измерения} --- Однозначно определены
        \item \textbf{Операция сложения, вычитания} --- Да
        \item \textbf{Операция деления, умножения} --- Да
        \item \textbf{Допустимое преобразование} --- Тождественное $f(x) = x$
        \item \textbf{Мода} --- Да
        \item \textbf{Медиана} --- Да
        \item \textbf{Ср. арифм и хар-ки рассеяния} --- Да
    \end{itemize}

    \textbf{Примеры: } Число предметов, событий

    \subsection{С чем будем работать на САНИ?}

    Ключевые факты:

    \begin{itemize}
        \item В основном дискретные генеральные совокупности (конечные случайные величины или векторы)
        \item Зачастую на вход нам подаются данные представленные в виде частот наблюдений, попавших в некоторые категории (или классы)
        \item Основная шкала измерения - номинальная
    \end{itemize}

    \begin{quote}
        \textbf{Основная задача САНИ} --- изучение связей, между различными качественными признаками многомерной генеральной совокупности
    \end{quote}

    \subsection{Проверка независимости двух дихотомических признаков}

    Обозначим, что это за признаки

    \begin{quote}
        \textbf{Дихотомические переменные} --- a.k.a бинарные, переменные, которые принимают всего два значения
    \end{quote}


    \begin{table}[h!]
    \centering
    \begin{tabular}{|c|c|c|c|}
      \hline
      & $Y_{1}$ & $Y_{2}$ & $p_{X}$ \\ \hline
      $X_{1}$ & $p_{11}$ & $p_{12}$ & $p_{1*} = p_{11} + p_{12}$  \\ \hline
      $X_{2}$ & $p_{21}$ & $p_{22}$ & $p_{2*} = p_{21} + p_{22}$ \\ \hline
      \textbf{$p_{Y}$} & $p_{*1} = p_{11} + p_{21}$ & $p_{*2} = p_{12} + p_{22}$ & $p_{**} = p_{11} + p_{12} + p_{21} + p_{22} = 1$ \\ \hline
    \end{tabular}
    \caption{Вероятностная таблица сопряженности}
    \label{tab:prob}
    \end{table}
    где $p_{ij}$ --- вероятность, того, что случайно взятый объект совокупности обладает категориями $X_{i}$ и $Y_{j}$

    \quad

    \begin{table}[h!]
    \centering
    \begin{tabular}{|c|c|c|c|}
      \hline
      & $Y_{1}$ & $Y_{2}$ & $p_{X}$ \\ \hline
      $X_{1}$ & $n_{11}$ & $n_{12}$ & $n_{1*} = n_{11} + n_{12}$  \\ \hline
      $X_{2}$ & $n_{21}$ & $n_{22}$ & $n_{2*} = n_{21} + n_{22}$ \\ \hline
      \textbf{$n_{Y}$} & $n_{*1} = n_{11} + n_{21}$ & $n_{*2} = n_{12} + n_{22}$ & $n_{**} = n_{11} + n_{12} + n_{21} + n_{22} = n_{\text{сумм.}}$ \\ \hline
    \end{tabular}
    \caption{Частотная таблица сопряженности}
    \label{tab:sample}
    \end{table}
    где $n_{ij}$ --- число элементов выборки, которое обладает категориями $X_{i}$ и $Y_{j}$

    \subsection{Независимость в таблицах сопряженности}

    Сформулируем нулевую гипотезу о независимости

    Изначально условие задается так:

    \[
    \begin{cases}
        p_{11} = p_{1*} * p_{*1} \\
        p_{12} = p_{1*} * p_{*2} \\
        p_{21} = p_{2*} * p_{*1} \\
        p_{22} = p_{2*} * p_{*2}
    \end{cases}
    \]

    \quad

    Заметим, что выражения полностью эквивалентны. Если выполняется одно из них, то выполняются все остальные. Оставим только одно выражение

    \quad


    \[
    \begin{aligned}
    & p_{11} = p_{1*} \cdot p_{*1} \\
    & \Rightarrow \\
    & p_{11} = (p_{11} + p_{12}) \cdot (p_{11} + p_{21}) \\
    & \Rightarrow \\
    & p_{11} = p_{11}^{2} + p_{11} \cdot p_{21} + p_{12} \cdot p_{11} + p_{12} \cdot p_{21} \\
    & \Rightarrow \\
    & p_{11}^{2} + p_{11} \cdot p_{21} + p_{12} \cdot p_{11} + p_{12} \cdot p_{21} - p_{11} = 0 \\
    & \Rightarrow \\
    & p_{11} \cdot (p_{11} + p_{21} + p_{12}) + p_{12} \cdot p_{21} - p_{11} = 0 \\
    & \Rightarrow \\
    & p_{11} \cdot (1 - p_{22}) + p_{12} \cdot p_{21} - p_{11} = 0 \\
    & \Rightarrow \\
    & p_{12} \cdot p_{21} - p_{11} \cdot p_{22} = 0
    \end{aligned}
    \]

    \quad

    Следовательно, критерием независимости является

    \begin{equation}
        p_{12} \cdot p_{21} = p_{11} \cdot p_{22}
    \end{equation}

    \subsubsection{МНП-оценка}

    Зададим функцию наибольшего правдоподобия

    \begin{equation}
        P(n_{11}, n_{12}, n_{21}, n_{22}) = \frac{(n_{11} + n_{12} + n_{21} + n_{22})!}{n_{11}! \cdot n_{12}! \cdot n_{21}! \cdot n_{22}!} \cdot p_{11}^{n_{11}} \cdot p_{12}^{n_{12}} \cdot p_{21}^{n_{21}} \cdot p_{22}^{n_{22}}
    \end{equation}
    где $n_{ij}$ --- наблюдаемое знач СВ, а $p_{ij}$ --- параметры распределения

    Найдем логарифмическую функцию правдоподобия

    \begin{multline}
        L(n_{11}, n_{12}, n_{21}, n_{22}) = \ln((n_{11} + n_{12} + n_{21} + n_{22})!) - \ln(n_{11}!) - \ln(n_{12}!) \\
        - \ln(n_{21}!) - \ln(n_{22}!) + n_{11} \cdot \ln(p_{11}) + n_{12} \cdot \ln(p_{12}) + n_{21} \cdot \ln(p_{21}) + n_{22} \cdot \ln(p_{22})
    \end{multline}

    Вычтем const для более удобной работы

    \begin{equation}
        L(n_{11}, n_{12}, n_{21}, n_{22}) = n_{11} \cdot \ln(p_{11}) + n_{12} \cdot \ln(p_{12}) + n_{21} \cdot \ln(p_{21}) + n_{22} \cdot \ln(p_{22})
    \end{equation}

    Найдем функцию Лагрнажа

    \begin{equation}
        \mathcal{L} = L + \lambda \cdot (1 - \sum_{i=1}^{2}{\sum_{j=1}^{2}{p_{ij}}})
    \end{equation}

    Следовательно:

    \[
    \begin{cases}
       \frac{n_{11}}{p_{11}} - \lambda = 0 \\
       \frac{n_{12}}{p_{12}} - \lambda = 0 \\
       \frac{n_{21}}{p_{21}} - \lambda = 0 \\
       \frac{n_{22}}{p_{22}} - \lambda = 0 \\
       \sum_{i=1}^{2}{\sum_{j=1}^{2}{p_{ij}}} = 1
    \end{cases}
    \]

    \quad

    \[
    \begin{cases}
       \frac{n_{11}}{p_{11}} = \frac{n_{12}}{p_{12}} = \frac{n_{21}}{p_{21}} = \frac{n_{22}}{p_{22}} \\
       \sum_{i=1}^{2}{\sum_{j=1}^{2}{p_{ij}}} = 1
    \end{cases}
    \]

    Следовательно:

    \begin{equation}
        \widehat{p_{ij}} = \frac{n_{ij}}{n_{11} + n_{12} + n_{21} + n_{22}}
    \end{equation}


    Несколько фактов:

    \begin{itemize}
        \item Безусловные оценки частот, по МНП и ММ совпадают и равны отношению частот $n_{ij}$ к общему объему выборки
        \item Безусловные оценки частот, являются несмещенными
        \item При больших объемах выборки и отсутствии малых частот, в соответствии с ЦПТ совместное распределение $n_{ij}$ стремится к многомерному нормальному распределению
    \end{itemize}

    \subsubsection{Случай независимости}

    Рассмотрим ситуацию при которой принимается гипотеза о независимости. Тогда:

    \begin{equation}
        \widehat{p_{ij}} = \frac{n_{i*} \cdot n_{*j}}{n_{**}^2}
    \end{equation}


    \subsubsection{Асимптотическая проверка независимости}

    \paragraph{Пирсон}

    \quad

    Для асимптотического тестирования используется критерий согласия $\chi^2$. Используемая статистика:

    \begin{equation}
        \chi^2_{набл} = \sum_{i=1}^{2}{\sum_{j=1}^{2}{ \frac{( n_{ij} - n_{ij}^{*} )^{2}}{ n_{ij}^{*} } }}
    \end{equation}
    где $n_{ij}^{*} = \frac{n_{i*} \cdot n_{*j}}{n_{**}}$
    
    \textbf{Крит. область}: $\chi^2_{\text{набл}} > \chi^2_{\text{кр}}$ где $\chi^2_{\text{кр}} =
    min\{x: P(\chi^2(1) > x) < a  \}$

    \quad

    \textbf{Почему берется 1 степень свободы?}

    \quad

    Начнем с того, что у нас 4 слагаемых и 4 параметра: $p_{11}, p_{12}, p_{21}, p_{22}$.
    Мы можем сразу заметить, что зная 3 параметра, мы точно находим 4, следовательно кол-во ``неизвестных`` параметров $<= 3$.
    Нужно заметить, что мы проверяем гипотезу принадлежности распределения вектора $(X, Y)$ к распределению, где $X$ и $Y$ независимы,
    а у такого распределения достаточно знать всего две $p_{ij}$ для того, чтобы вывести все остальные $p$ - шки.
    $\Rightarrow$ у нас получается $4 - 2 - 1 = 1$ степеней свободы. Ч.Т.Д

    \quad

    При попадении в критическую область нулевая гипотеза отвергается с вероятностью ошибки $a$

    \quad

    Рассмотрим вывод формулы критерия согласия, для этого возьмем отклонение суммы частот за вычетом 1, домноженное на $n_{**}$

    \quad

    Нам нужно привести формулу, к такому виду:

    \quad

    \begin{equation}
        n_{**} \cdot (\sum_{i=1}^{2} { \sum_{j=1}^{2}{\frac{n_{ij}^{2}}{n_{i*} \cdot n_{*j} }}} - 1) \\
    \end{equation}

    \quad

    Приступим:

    \quad

    \[
    \begin{gathered}
         \sum_{i=1}^{2}{\sum_{j=1}^{2}{ \frac{( n_{ij} - \frac{n_{i*} \cdot n_{*j}}{n_{**}} )^{2} \cdot n_{**}}{n_{i*} \cdot n_{*j} } } } \\
         = \sum_{i=1}^{2}{\sum_{j=1}^{2}{ \frac{( \frac{n_{ij} \cdot n_{**} - n_{i*} \cdot n_{*j}}{n_{**}} )^{2} \cdot n_{**}}{n_{i*} \cdot n_{*j} } } } \\
         = \sum_{i=1}^{2}{\sum_{j=1}^{2}{ \frac{( n_{ij} \cdot n_{**} - n_{i*} \cdot n_{*j} )^{2} }{n_{i*} \cdot n_{*j} \cdot n_{**} } } } \\
         = \sum_{i=1}^{2}{\sum_{j=1}^{2}{  \frac{(n_{ij} \cdot n_{**})^{2}}{ n_{i*} \cdot n_{*j} \cdot n_{**} }  }} \\
         - 2 \cdot \sum_{i=1}^{2}{\sum_{j=1}^{2}{   \frac{(n_{i*} \cdot n_{*j}) \cdot (n_{ij} \cdot n_{**})}{ n_{i*} \cdot n_{*j} \cdot n_{**}}  }} \\
         + \sum_{i=1}^{2}{\sum_{j=1}^{2}{   \frac{(n_{i*} \cdot n_{*j})^{2}}{n_{i*} \cdot n_{*j} \cdot n_{**}}  }} \\
         = \sum_{i=1}^{2}{\sum_{j=1}^{2}{  \frac{n_{ij}^{2} \cdot n_{**}}{ n_{i*} \cdot n_{*j} }  }} \\
         - 2 \cdot n_{**} \\
         + \sum_{i=1}^{2}{\sum_{j=1}^{2}{   \frac{n_{i*} \cdot n_{*j}}{ n_{**}}  }}
    \end{gathered}
    \]

    \quad

    Рассмотрим, последнее слагаемое:

    \quad

    \[
    \begin{gathered}
        \frac{n_{1*} \cdot n_{*1} + n_{1*} \cdot n_{*2} + n_{2*} \cdot n_{*1} + n_{2*} \cdot n_{*2}}{ n_{**}} \\
         = \frac{n_{1*} \cdot (n_{*1} + n_{*2}) + n_{2*} \cdot (n_{*1} + n_{*2})}{ n_{**}} \\
         = \frac{n_{**}^{2}}{n_{**}} \\
         = n_{**}
    \end{gathered}
    \]

    \quad

    Следовательно, наше выражение имеет вид:

    \quad

    \[
    \begin{gathered}
        \sum_{i=1}^{2}{\sum_{j=1}^{2}{  \frac{n_{ij}^{2} \cdot n_{**}}{ n_{i*} \cdot n_{*j} }  }} \\
         - 2 \cdot n_{**} \\
         + n_{**} \\
         = \sum_{i=1}^{2}{\sum_{j=1}^{2}{  \frac{n_{ij}^{2} \cdot n_{**}}{ n_{i*} \cdot n_{*j} }  }} - n_{**} \\
         = n_{**} \cdot (\sum_{i=1}^{2}{\sum_{j=1}^{2}{  \frac{n_{ij}^{2} }{ n_{i*} \cdot n_{*j} }  }} - 1)
    \end{gathered}
    \]
    
    \textbf{Ч.Т.Д}


    Рассмотрим еще одно выражение:

    \begin{equation}
        \chi^{2}_{\text{набл}} = \frac{n_{**} \cdot (n_{11}
        \cdot n_{22} - n_{21} \cdot n_{12})^{2}}{n_{1*} \cdot n_{*1} \cdot n_{2*} \cdot n_{*2}}
    \end{equation}

    Покажем, что данное выражение равносильно предыдущему.
    Рассмотрим предыдущее выражение, поделенное на $n_{**}$
    \[
    \begin{gathered}
        \sum_{i=1}^{2}{\sum_{j=1}^{2}{  \frac{n_{ij}^{2} }{ n_{i*} \cdot n_{*j} }  }} - 1 \\
        = \frac{n_{11}^{2} }{n_{1*} \cdot n_{*1}} + \frac{n_{12}^{2} }{n_{1*} \cdot n_{*2}} \\
        + \frac{n_{21}^{2} }{n_{2*} \cdot n_{*1}} + \frac{n_{22}^{2} }{n_{2*} \cdot n_{*2}} - 1 \\
        = \frac{ n_{11}^{2} \cdot n_{2*} \cdot n_{*2} }{n_{1*} \cdot n_{*1} \cdot n_{2*} \cdot n_{*2}} +
        \frac{n_{12}^{2} \cdot n_{2*} \cdot n_{*1} }{n_{1*} \cdot n_{*1} \cdot n_{2*} \cdot n_{*2}} \\
        + \frac{n_{21}^{2} \cdot n_{1*} \cdot n_{*2} }{n_{1*} \cdot n_{*1} \cdot n_{2*} \cdot n_{*2}} +
        \frac{n_{22}^{2} \cdot n_{1*} \cdot n_{*1} }{n_{1*} \cdot n_{*1} \cdot n_{2*} \cdot n_{*2}} - 1 \\
        = \frac{n_{11}^{2} \cdot n_{2*} \cdot n_{*2} + n_{12}^{2} \cdot n_{2*} \cdot n_{*1} + n_{21}^{2} \cdot n_{1*} \cdot n_{*2} + n_{22}^{2} \cdot n_{1*} \cdot n_{*1}}
        {n_{1*} \cdot n_{*1} \cdot n_{2*} \cdot n_{*2}} \\ - 1
    \end{gathered}
    \]

    Рассмотрим числитель отдельно, занеся туда 1


    \begin{multline*}
       \\ n_{11}^{2} \cdot n_{2*} \cdot n_{*2} + n_{12}^{2} \cdot n_{2*} \cdot n_{*1}
       + n_{21}^{2} \cdot n_{1*} \cdot n_{*2} + n_{22}^{2} \cdot n_{1*} \cdot n_{*1} \\
       - n_{1*} \cdot n_{*1} \cdot n_{2*} \cdot n_{*2} \\
       = n_{11}^{2} \cdot (n_{21} + n_{22}) \cdot (n_{12} + n_{22}) \\
       + n_{12}^{2} \cdot (n_{21} + n_{22}) \cdot (n_{11} + n_{21}) \\
       + n_{21}^{2} \cdot (n_{11} + n_{12}) \cdot (n_{12} + n_{22}) \\
       + n_{22}^{2} \cdot (n_{11} + n_{12}) \cdot (n_{11} + n_{21}) \\
       - n_{1*} \cdot n_{*1} \cdot n_{2*} \cdot n_{*2} \\
       = n_{11}^{2} \cdot (n_{21} \cdot n_{12} + n_{21} \cdot n_{22} + n_{22} \cdot n_{12} + n_{22}^{2}) \\
       + n_{12}^{2} \cdot (n_{21} \cdot n_{11} + n_{21}^{2} + n_{22} \cdot n_{11} + n_{22} \cdot n_{21}) \\
       + n_{21}^{2} \cdot (n_{11} \cdot n_{12} + n_{11} \cdot n_{22} + n_{12}^{2} + n_{12} \cdot n_{22}) \\
       + n_{22}^{2} \cdot (n_{11}^{2} + n_{11} \cdot n_{21} + n_{12} \cdot n_{11} + n_{12} \cdot n_{21}) \\
       - (n_{11} + n_{12}) \cdot (n_{11} + n_{21}) \cdot (n_{21} + n_{22}) \cdot (n_{12} + n_{22}) \\
       = n_{11}^{2} \cdot (n_{21} \cdot n_{12} + n_{21} \cdot n_{22} + n_{22} \cdot n_{12} + n_{22}^{2}) \\
       + n_{12}^{2} \cdot (n_{21} \cdot n_{11} + n_{21}^{2} + n_{22} \cdot n_{11} + n_{22} \cdot n_{21}) \\
       + n_{21}^{2} \cdot (n_{11} \cdot n_{12} + n_{11} \cdot n_{22} + n_{12}^{2} + n_{12} \cdot n_{22}) \\
       + n_{22}^{2} \cdot (n_{11}^{2} + n_{11} \cdot n_{21} + n_{12} \cdot n_{11} + n_{12} \cdot n_{21}) \\
       - (n_{11}^{2} + n_{11} \cdot n_{21} + n_{12} \cdot n_{11} + n_{12} \cdot n_{21}) \\
       \cdot (n_{21} \cdot n_{12} + n_{21} \cdot n_{22} + n_{22} \cdot n_{12} + n_{22}^{2}) \\
       = n_{12}^{2} \cdot (n_{21} \cdot n_{11} + n_{21}^{2} + n_{22} \cdot n_{11} + n_{22} \cdot n_{21}) \\
       + n_{21}^{2} \cdot (n_{11} \cdot n_{12} + n_{11} \cdot n_{22} + n_{12}^{2} + n_{12} \cdot n_{22}) \\
       + n_{22}^{2} \cdot (n_{11}^{2} + n_{11} \cdot n_{21} + n_{12} \cdot n_{11} + n_{12} \cdot n_{21}) \\
       - (n_{11} \cdot n_{21} + n_{12} \cdot n_{11} + n_{12} \cdot n_{21}) \\
       \cdot (n_{21} \cdot n_{12} + n_{21} \cdot n_{22} + n_{22} \cdot n_{12} + n_{22}^{2}) \\
       = n_{12}^{2} \cdot (n_{21} \cdot n_{11} + n_{21}^{2} + n_{22} \cdot n_{11} + n_{22} \cdot n_{21}) \\
       + n_{21}^{2} \cdot (n_{11} \cdot n_{12} + n_{11} \cdot n_{22} + n_{12}^{2} + n_{12} \cdot n_{22}) \\
       + n_{22}^{2} \cdot n_{11}^{2} \\
       - (n_{11} \cdot n_{21} + n_{12} \cdot n_{11} + n_{12} \cdot n_{21}) \\
       \cdot (n_{21} \cdot n_{12} + n_{21} \cdot n_{22} + n_{22} \cdot n_{12}) \\
       = n_{12}^{2} \cdot (n_{21} \cdot n_{11} + n_{21}^{2} + n_{22} \cdot n_{11} + n_{22} \cdot n_{21}) \\
       + n_{21}^{2} \cdot (n_{11} \cdot n_{12} + n_{11} \cdot n_{22} + n_{12}^{2} + n_{12} \cdot n_{22}) \\
       + n_{22}^{2} \cdot n_{11}^{2} \\
       - n_{11} \cdot n_{21}^2 \cdot n_{12} - n_{11} \cdot n_{21}^{2} \cdot n_{22} - n_{11} \cdot n_{21} \cdot n_{22} \cdot n_{12} \\
       - n_{12}^{2} \cdot n_{11} \cdot n_{21} - n_{12} \cdot n_{11} \cdot n_{21} \cdot n_{22} - n_{12}^{2} \cdot n_{11} \cdot n_{22} \\
       -  n_{12}^{2} \cdot n_{21}^{2} - n_{12} \cdot n_{21}^{2} \cdot n_{22} - n_{22} \cdot n_{12}^{2} \cdot n_{21} \\
       = n_{12}^{2} \cdot (n_{22} \cdot n_{11} + n_{22} \cdot n_{21}) \\
       + n_{21}^{2} \cdot (n_{11} \cdot n_{12} + n_{11} \cdot n_{22} + n_{12}^{2} + n_{12} \cdot n_{22}) \\
       + n_{22}^{2} \cdot n_{11}^{2} \\
       - n_{11} \cdot n_{21}^2 \cdot n_{12} - n_{11} \cdot n_{21}^{2} \cdot n_{22} - n_{11} \cdot n_{21} \cdot n_{22} \cdot n_{12} \\
       - n_{12} \cdot n_{11} \cdot n_{21} \cdot n_{22} - n_{12}^{2} \cdot n_{11} \cdot n_{22} \\
       - n_{12} \cdot n_{21}^{2} \cdot n_{22} - n_{22} \cdot n_{12}^{2} \cdot n_{21} \\
       = n_{12}^{2} \cdot n_{22} \cdot n_{11} \\
       + n_{21}^{2} \cdot (n_{11} \cdot n_{22} + n_{12}^{2} + n_{12} \cdot n_{22}) \\
       + n_{22}^{2} \cdot n_{11}^{2} \\
       - n_{11} \cdot n_{21}^{2} \cdot n_{22} - n_{11} \cdot n_{21} \cdot n_{22} \cdot n_{12} \\
       - n_{12} \cdot n_{11} \cdot n_{21} \cdot n_{22} - n_{12}^{2} \cdot n_{11} \cdot n_{22} \\
       - n_{12} \cdot n_{21}^{2} \cdot n_{22}  \\
       = n_{12}^{2} \cdot n_{22} \cdot n_{11} \\
       + n_{21}^{2} \cdot n_{12}^{2} + n_{22}^{2} \cdot n_{11}^{2} \\
       - n_{11} \cdot n_{21} \cdot n_{22} \cdot n_{12} \\
       - n_{12} \cdot n_{11} \cdot n_{21} \cdot n_{22} - n_{12}^{2} \cdot n_{11} \cdot n_{22} \\
       = n_{21}^{2} \cdot n_{12}^{2} + n_{22}^{2} \cdot n_{11}^{2} \\
       - n_{11} \cdot n_{21} \cdot n_{22} \cdot n_{12} \\
       - n_{12} \cdot n_{11} \cdot n_{21} \cdot n_{22} \\
       = (n_{21} \cdot n_{12} - n_{22} \cdot n_{11})^{2} \\
    \end{multline*}
    
    
    Добавим знаменатель и $n_{**}$, тогда наша изначальная формула равносильна 
    
    \begin{equation}
        \frac{n_{**} \cdot (n_{11}
        \cdot n_{22} - n_{21} \cdot n_{12})^{2}}{n_{1*} \cdot n_{*1} \cdot n_{2*} \cdot n_{*2}}
    \end{equation}
    
    \textbf{Ч.Т.Д}
    

    \paragraph{Критерий отношения правдоподобия}

    \quad

    Рассмотрим данный критерий в общем виде, он задается по такому принципу:
    Путь у нас есть модель c пространством параметров $\Theta$.
    $H_{0}$ задается так: пусть параметр $\theta \in \Theta_{0}$ где $\Theta_{0} \subset \Theta$.
    При этом $H_{1}$ гласит: $\theta \in \Theta \textbackslash \Theta_{0}$

    \quad

    Тестовая статистика задается как:

    \quad

    \begin{equation}
        \lambda = - 2 \cdot \ln(  \frac{\sup_{\theta \in \Theta_{0}} L(\theta | x) }{\sup_{\theta \in \Theta} L(\theta | x) } )
    \end{equation}

    легко заметить, что

    \[
    \begin{gathered}
        \sup_{\theta \in \Theta_{0}} L(\theta | x) <= \sup_{\theta \in \Theta} L(\theta | x) \\
        \Rightarrow \\
        \frac{\sup_{\theta \in \Theta_{0}} L(\theta | x) }{\sup_{\theta \in \Theta} L(\theta | x) } <= 1 \\
        \Rightarrow \\
        \ln(  \frac{\sup_{\theta \in \Theta_{0}} L(\theta | x) }{\sup_{\theta \in \Theta} L(\theta | x) } ) <= 0 \\
        \Rightarrow \\
        \lambda >= 0
    \end{gathered}
    \]

    Попробуем определить распределение данной статистики.
    Для начала представим нашу функцию по другому
    \begin{equation}
        \lambda = -2 \cdot (\ln(\sup_{\theta \in \Theta_{0}} L(\theta | x)) - \ln(\sup_{\theta \in \Theta} L(\theta | x)) )
    \end{equation}

    Пусть $\ell(\theta) = \ln(L(\theta | x))$

    Аппроксимируем функцию разложением в ряд Тейлора вокруг точки максимизирующей функцию на всем пространстве параметров для знаменателя (обозначим точку как $\theta^{*}$), тогда:

    \begin{equation}
        \ell(\theta) \approx \ell(\theta^{*}) + \frac{\partial \ell(\theta^{*})}{\partial \theta} \cdot (\theta - \theta^{*}) +
        0.5 \cdot (\theta - \theta^{*})^{T} \cdot \frac{\partial^{2} \ell(\theta^{*})}{\partial \theta^{2}} \cdot (\theta - \theta^{*})
    \end{equation}

    \quad

    Так как, это точка максимума то первый дифференциал равен нулевому вектору, следовательно

    \quad

    \begin{equation}
        \ell(\theta) \approx \ell(\theta^{*}) +
        0.5 \cdot (\theta - \theta^{*})^{T} \cdot \frac{\partial^{2} \ell(\theta^{*})}{\partial \theta^{2}} \cdot (\theta - \theta^{*})
    \end{equation}

    \quad

    \textbf{Тут же всплывает первое предположение, что точка глобального максимума на пространстве парметров не является краевой}

    \quad

    Пусть $\theta^{**}$ - точка в которой максимизируется функция на $\Theta_{0}$

    Подставим в это разложение $\theta^{**}$ и взглянем на нашу статистику

    \begin{equation}
        \lambda = -2 \cdot (\ell(\theta^{*}) +
        0.5 \cdot (\theta^{**} - \theta^{*})^{T} \cdot \frac{\partial^{2} \ell(\theta^{*})}{\partial \theta^{2}} \cdot (\theta^{**} - \theta^{*}) - \ell(\theta^{*}))
    \end{equation}

    \quad

    В итоге:

    \quad

    \begin{equation}
         \lambda = -(\theta^{**} - \theta^{*})^{T} \cdot \frac{\partial^{2} \ell(\theta^{*})}{\partial \theta^{2}} \cdot (\theta^{**} - \theta^{*})
    \end{equation}

    Рассмотрим поведение данного параметра при $n \rightarrow \inf$ и верной $H_{0}$.
    Заметим, что тогда $\theta^{**} \rightarrow \theta^{*} \rightarrow \theta_{true}$ т.е оба параметра стремятся к одному истинному параметру.
    При этом матрица $\frac{\partial^{2} \ell(\theta^{*})}{\partial \theta^{2}}$  стремится к матрице Фишера (Информация Фишера для многомерных данных --- матрица, а не вектор).
    Каждая из оценок $\theta^{**}$ и $\theta^{*}$ является асимптотически нормальной, по свойству оценки МНП.
    Следовательно разница данных величин имеет нормальное распределение

    \quad

    Далее в дело вступает следующая теорема:

    \begin{quote}

    Если $Z$ --- многомерная нормально распределенная случайная величина с нулевым средним и ковариационной матрицей $B$,
    то квадратичная форма $Z^{T} A Z$ имеет распределение $\chi^{2}$ с числом степеней свободы, равным рангу матрицы $A$,
    при условии, что

    1) $AB$ --- идемпотентная матрица (т.е. $A B A B= A B$)

    2) $A$ симметричная матрица

    3) $A$ имеет собственные значения $0$ и $1$.

    \end{quote}

    \quad

    \textbf{Доказательство}

    \quad

    \begin{quote}
        По свойству многомерного нормального распределения,
        если $X \sim \mathcal{N}(\mu, B)$ и $\dim(X) = n$,
        тогда $AX \sim \mathcal{N}(A \mu, A B A^{T})$ где матрица $A$ имеет размерность $m$ на $n$.
        Нам нужно привести наше выражение к виду $Z^{T} C Z$ где $Z \sim \mathcal{N}(\bar{0}, \operatorname{E}) $,
        а $C$ --- матрица с 1 на диагонали.

        Пусть нам дан случайный вектор $Z \sim \mathcal{N}(\mu, B)$

        Так как ковариационная матрица симметрична, существует матрица $K$ такая, что $K B K^{T} = \operatorname{E}$.
        Данная матрица $K$ обязательно обратима, следовательно $Z$ может быть представлена как $Z = K^{-1} Z_{*}$, где
        $Z_{*} \sim \mathcal{N}(K \mu , \operatorname{E})$.
        Следовательно $Z^{T} A Z = (K^{-1} Z_{*})^{T} A K^{-1} Z_{*} = Z_{*}^{T} {K^{-1}}^{T}A K^{-1} Z_{*}$

        Представим $Z_{*}$ как $Z_{*} = Z_{**} + K \mu$, где $Z_{**} \sim \mathcal{N}(\bar{0} , \operatorname{E})$.
        Следовательно, наше выражение равно $(Z_{**} + K \mu)^{T} {K^{-1}}^{T}A K^{-1} (Z_{**} + K \mu)$.
        $(Z_{**} + K \mu)^{T} {K^{-1}}^{T}A K^{-1} Z_{**} + (Z_{**} + K \mu)^{T} {K^{-1}}^{T}A K^{-1} K \mu =
        (Z_{**} + K \mu)^{T} {K^{-1}}^{T}A K^{-1} Z_{**} + (Z_{**} + K \mu)^{T} {K^{-1}}^{T}A \mu$

        Раскрывая первую скобку:

        $(Z_{**} + K \mu)^{T} {K^{-1}}^{T}A K^{-1} Z_{**} + (Z_{**} + K \mu)^{T} {K^{-1}}^{T}A \mu =
        Z_{**}^{T} {K^{-1}}^{T}A K^{-1} Z_{**} + \mu^{T} K^{T} {K^{-1}}^{T}A K^{-1} Z_{**} +
        Z_{**}^{T} {K^{-1}}^{T}A \mu   + \mu^{T} K^{T} {K^{-1}}^{T} A \mu$

        Сократим, часть компонентов как единичные матрицы

        $Z_{**}^{T} {K^{-1}}^{T}A K^{-1} Z_{**} + \mu^{T} A K^{-1} Z_{**} +
        Z_{**}^{T} {K^{-1}}^{T}A \mu  + \mu^{T} A \mu$

        Заметим, что второе и третье слагаемое равны, так как на выходе мы имеем матрицу с размерностью $1$ на $1$, а эти слагаемые транспонированные версии друг друга.

        Тогда наше выражение представимо как:

        $Z_{**}^{T} {K^{-1}}^{T}A K^{-1} Z_{**} + 2 \cdot \mu^{T} A K^{-1} Z_{**} + \mu^{T} A \mu$





    \end{quote}




    \quad

    Рассмотрим как выглядят вектора $\theta^{**}$ и $\theta^{*}$ с которыми мы работаем.
    При подстановке $\dim(\theta^{**}) = \dim(\theta^{*}) = n$, однако допустим что на самом деле
    $\theta^{**}$ --- линейное отображение из подпространства с меньшей размерностью (так как наложены ограничения)

    \quad

    $\theta^{**} \sim \mathcal{N}(\theta_{true}, \frac{ C \mathcal{I}^{-1}( \bar{\theta} )  C^{T} }{n})$, а второй вектор
    $\theta^{*} \sim \mathcal{N}(\theta_{true}, \frac{\mathcal{I}^{-1}( \theta )}{n})$

    Рассмотрим распределение разницы векторов:

    $\theta^{**} - \theta^{*} \sim \mathcal{N}(0, \operatorname{Cov}(\theta^{**} + \theta^{*}))$

    где

    $\operatorname{Cov}(\theta^{**} + \theta^{*}) = \frac{ C \mathcal{I}^{-1}( \bar{\theta} )  C^{T} }{n} + \frac{\mathcal{I}^{-1}( \theta )}{n} + 2 \cdot \operatorname{Cov}(\theta^{**}, \theta^{*})$
    В предельном случае, оценки являются независимыми (i.e они асимптотически независимы), следовательно

    $\operatorname{Cov}(\theta^{**} + \theta^{*}) = \frac{ C \mathcal{I}^{-1}( \bar{\theta} )  C^{T} }{n} + \frac{\mathcal{I}^{-1}( \theta )}{n}$



    \quad


    Рассмотрим статистику критерия:

    \begin{equation}
        \chi^{2}_{\text{инф}} = 2 \cdot \sum_{i=1}^{2} {\sum_{j=1}^{2} {n_{ij} \cdot \ln(\frac{n_{ij}}{n_{ij}^{*}}) } }
    \end{equation}
    где $n_{ij}^{*} = \frac{n_{i*} \cdot n_{*j}}{n_{**}}$ получается, что $n_{ij}^{*}$ --- теоретическая частота

    
    \subsubsection{Точная проверка независимости}

    Гипотеза независимости $H0: p_{11} \cdot p_{22} = p_{12} \cdot p_{21}$

    \textbf{Крит. область:} Такая же как и у прошлого теста





\end{document}